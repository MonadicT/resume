\documentclass[a4paper, 13pt,line]{article}

\usepackage{xcolor,colortbl}
\usepackage{tabularx}
\usepackage{makecell}
\usepackage{float}
\usepackage{wasysym}
\usepackage{marvosym}
\usepackage[alpine]{ifsym}

\usepackage[margin=0.5in]{geometry}

\definecolor{Gray}{gray}{0.95}

\begin{document}
\noindent
\begin{tabular}[h]{lcr}
&{\Huge \textbf{Praki Prakash}} & \\
\\
& {\Letter: {praki.prakash@gmail.com}} & \\
\\
\\
\raisebox{-1pt}\FilledHut\space San Carlos, CA 94070 & & \phone\space 650-200-5155\\
\\
\Xhline{2.5\arrayrulewidth}
\hline\hline\Xhline{1.5\arrayrulewidth}
\rowcolor{Gray}&&\\
\rowcolor{Gray}& CHIEF TECHNOLOGY OFFICER &\\
\rowcolor{Gray}
& System Architecture \raisebox{1pt}\textbullet{} Technology
Direction \raisebox{1pt}\textbullet{} Software Development &\\
\rowcolor{Gray}&&\\
\Xhline{1.5\arrayrulewidth}
\end{tabular}

\bigskip
\noindent
Entrepreneurial technologist, with more than two decades of experience
in designing and building innovative products. With a passion for
building simple and robust software to meet the needs of the users,
combines a broad range of experience with a can-do
attitude. Identifies technology options and makes engineering
trade-offs that allow meeting immediate needs with a well-defined
direction for evolution.
\bigskip

\begin{table}[!ht]
{\renewcommand{\arraystretch}{1.9}
\begin{tabularx}{\textwidth}{XcX}
\hline\hline\Xhline{2.5\arrayrulewidth}
\rowcolor{Gray}      &PROFESSIONAL EXPERIENCE& \\
\Xhline{2.5\arrayrulewidth}
\hline
\end{tabularx}
}
\end{table}

\begin{table}[!ht]
\begin{tabularx}{\textwidth}{lXr}
{\large \textbf Chief Architect, Picarro Inc, Santa Clara, CA} & & Jan
2015 - Present\\
\hline
\Xcline{1-1}{1.5pt}\\
\end{tabularx}
\end{table}
\vspace{-15pt}
\noindent As the Chief Architect of a pioneering technology company that is redefining the natural gas
distribution industry, I steer the architecture of all software
developed in the company and provide technology vision and direction
needed to transform the company with a scientific instrumentation
background to a data analytics company.

\bigskip
\begin{itemize}
\item Designed and implemented a ReactNative Android app for gas leak
  detection using a portable analyzer. Uses heatmap overlaid on Google
  maps to aid user in finding and recording leak locations quickly.
\item Realigned the architecture of Picarro's flagship product
  Surveyor from a monolithic mess to a horizontally scalable
  micro-service architecture leading to shorter development and
  test cycles, improved customer satisfaction.
\item Introduced the team to cluster computing technology, Apache
  Spark and scalable databases, Crate.IO, enabling key R\&D
  efforts.
\item Defined a three-year technology roadmap culminating in a modern
  and efficient datacenter cluster managed by Apache Mesos.
\item Defined and led Enterprise Readiness initiatives. Addressed
  security concerns about sensitive customer data with a cloud-based
  analytics workshop which is securely managed and fully
  audited. Guided performance and scalability improvements.
\item Defined and building a Windows to Linux strategy for Picarro
  analyzers to enable better deployment (Docker), simplified
  communication (HTTP2, Message bus) and ReactJS-based UI.
\item Assisted various teams in the company with their technology
  selection process, mentored new employees in software development
  process. Grew awareness of software engineering best practices in
  hardware and system-centric teams.
\item Revamped PKI infrastructure with certificates provisioned from
  LetsEncrypt.org. Every Picarro analyzer in the field is provisioned
  individual X509 certificates enabling cryptographically secure
  communication.
\end{itemize}

\bigskip\noindent Some of my hands-on contributions:
\begin{itemize}
\item Picarro App \textemdash Android app built with React Native,
  Redux, Redux Saga and Bing maps.
\item A Simple and scalabale data lake built using Crate.IO for data
  analytics which enabled a time-critical analysis to be done on a
  scale considered undoable.
\item Survey Drive Simulator - a highly flexible and scalable tool
  which provides the capability to simulate any condition needed to
  test Surveyor and eliminates lengthy and expensive driving tests.
\item Surveyor Dashboard, a business metrics dashboard built using
  ReactJS and deployed using Docker.
\end{itemize}

\noindent Skills: Apache Spark, Crate, Docker, ReactJS, React Native,
React Redux, React Saga, D3, GDAL, Java, Python, JavaScript, Groovy.

\newpage
\begin{table}[!htbp]
\begin{tabularx}{\textwidth}{lXr}
{\large \textbf Master Technologist, HP Inc, Sunnyvale, CA} & & Mar 2014 - Jan 2015\\
\hline
\Xcline{1-1}{1.5pt}\\
\end{tabularx}
\end{table}
\vspace{-15pt}

\noindent Performance and Scalability lead for HP/Arcsight's flagship
product Enterprise Security Manager. Defined the performance metrics
for a comprehensive performance and scalability study of ESM,
identified bottlenecks. Designed a distributed query processor to
scale beyond the current single-threaded approach.

\bigskip
\noindent Presented ESM performance study results at Protect 2014 conference.

\noindent Skills: Scalability, query parallelization, performance measurement
and analysis, grrovy, R, Java

\begin{table}[!htbp]
\begin{tabularx}{\textwidth}{lXr}
{\large \textbf Founder, GuidingWave Technology, San Carlos, CA} & &
Jan 2010 - Dec 2013\\
\hline
\Xcline{1-1}{1.5pt}\\
\end{tabularx}
\end{table}

\noindent Conceived, designed and developed a Data Center Management
tool for use with VMware vSphere, with the innovative idea of using
Machine Learning to understand datacenters and provide expert guidance
in their management. A data gathering agent monitors all changes to the
data center, collects performance data and events. This data is stored
in a purpose-built database and a data analysis module utilizing
machine learning algorithms to learn the capacity and utilization
patterns.

\bigskip\noindent The user interface component provides widgets to
visualize the data from the Graph database and build custom dashboards
in a browser-based designer.

\bigskip\noindent Skills: Clojure, vSphere, ESX/ESXi, Sketch-based
data structures, Wavelets

\begin{table}[!ht]
\begin{tabularx}{\textwidth}{lXr}
{\large \textbf Staff Engineer, VMware Inc, Palo Alto, CA} & &
July 2005 - Dec 2010\\
\hline
\Xcline{1-1}{1.5pt}\\
\end{tabularx}
\end{table}
\vspace{-15pt}

\noindent Lead Architect for VMware's Capacity Planner team.

\begin{itemize}
\item Architect for VMware Capacity Planner team overseeing several
  versions of capacity planning products at VMware.
\item Provided technical guidance to a team of more than thirty
  engineers involved in building and maintaining VMware Capacity
  Planner product and service.
\item Designed and implemented a Virtual Machine to Host mapping
  algorithm.
\item Redesigned VMware Capacity Planner service to scale to large
  numbers of concurrent users, receive and process in excess of one
  terabyte of inventory and performance data from capacity planning
  assessments at customer sites. The redesign virtually eliminated
  failures under load and reduced processing lag times from days to
  hours.
\item Prototyped an operations dashboard for VMware Capacity Planner
  service to measure and monitor critical aspects of the service.
\end{itemize}

\begin{table}[!ht]
\begin{tabularx}{\textwidth}{lXr}
{\large Engineering Manager, Yahoo! \hspace{-1mm} Personals, Sunnyvale, CA} & &
Nov 2004 - July 2005\\
\hline
\Xcline{1-1}{1.5pt}\\
\end{tabularx}
\end{table}
\vspace{-15pt}

\noindent Managed one of Yahoo!'s most profitable properties, Yahoo! Personals
search engine team of ten engineers.

\begin{itemize}
\item Managed Yahoo!\hspace{-1mm} Personals search team with both
  developmental and operational responsibilities for the popular
  Yahoo! service.
\item Successfully implemented a Business Continuity Planning solution
  for Yahoo!\hspace{-1mm} Personals profile database eliminating
  single point of failure and significantly improving the failure and
  disaster recovery aspects of the service.
\item Delivered new features and improvements to profile matching
  periodically.
\end{itemize}

\begin{table}[!ht]
\begin{tabularx}{\textwidth}{lXr}
{\large Founding Engineer, Westbridge Technologies, Inc., Mountain View, CA} & &
Oct 2001 - Nov 2004\\
\hline
\Xcline{1-1}{1.5pt}\\
\end{tabularx}
\end{table}
\vspace{-15pt}

\noindent As the first employee of a XML technology startup, set the
architecture and technology direction for the company's innovative XML
messaging product.

\begin{itemize}
\item Joined Westbridge Technologies as its first employee and played
  a guiding role in the design and implementation of XML Message
  Server product.
\item Mentored the local and offshore engineering teams.
\item Designed and implemented a metadata-based server core which
  proved to be highly flexible and reusable and resulted in
  significant productivity boosts.
\item Designed and implemented XML Signature and Encryption modules.
\item Provided quick prototypes of features and functionality leading
  to many customer acquisitions.
\end{itemize}

\begin{table}[!ht]
\begin{tabularx}{\textwidth}{lXr}
{\large Sr. Director of Engineering, Arzoo, Inc., Fremont, CA} & &
Mar 2000 - July 2001\\
\hline
\Xcline{1-1}{1.5pt}\\
\end{tabularx}
\end{table}
\vspace{-15pt}

\begin{itemize}
\item Joined as a Director of Engineering and soon entrusted with
responsibilities for all development teams of more than sixty engineers.
\item Hired and mentored technical leaders and engineers.
\item Introduced Java and Servlets to the company.
\item Delivered a new web-based service meeting strict deadlines.
\end{itemize}

\begin{table}[H]
{\renewcommand{\arraystretch}{1.9}
\begin{tabularx}{\textwidth}{XcX}
\hline\hline\Xhline{2.5\arrayrulewidth}
\rowcolor{Gray}      &EDUCATION& \\
\Xhline{2.5\arrayrulewidth}
\hline
\end{tabularx}
}
\end{table}

\noindent\textbf{Indian Institute of Technology}, Chennai, India \hfill \textbf{Master of Technology}\vspace{1mm}\\
\textbf{University Visvesvaraya College of Engineering}, Bengaluru, India \hfill \textbf{Bachelor of Engineering}\vspace{-1.5mm}\\%\vspace{-1mm}%

\end{document}
